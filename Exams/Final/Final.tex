\documentclass[]{exam}
\usepackage[utf8]{inputenc}
\usepackage{amsmath,amssymb}
\usepackage{multicol}
\usepackage[a4paper,width=150mm,top=25mm,bottom=25mm]{geometry}
\usepackage{graphicx}
\title{Innopolis University \\ Analytical Geometry and Linear Algebra~I
\\ Final Exam}
\date{December 14, 2022.}

\begin{document}


% 
% \maketitle


\addpoints
%\gradetable[h][questions]
\
\centering
Analytical Geometry and Linear Algebra~I. Final Exam. December 14, 2022

\textbf{VARIANT 1}
\bigskip

\begin{tabular}{cc}
    \begin{tabular}[b]{|p{11cm}|c|}
    \hline
    Full name: & Group: \\ \hline
     &  \\ 
     & \\
     \hline
    \end{tabular}
\\

    % \begin{tabular}[b]{|c|c|c|c|c|c|c|c|c|c|c|c|}
    % \hline
    % Task: & 1 & 2 & 3 & 4 & 5 & 6 & 7 &8 & 9 & 10 & Total \\ \hline

    % Score: &  &  &  &  &  & &  & & &  & \_\_\_\_\_ of 30 pts. \\ \hline
    % \end{tabular}
\end{tabular}

\begin{flushleft}
    % \textbf{\Large{Rules:}}
    % \begin{enumerate}
    \textbf{In each sheet}, you \textbf{should} write your last name, first name, variant number, and group number in the \textbf{upper right} corner. Unsigned sheets or sheets without the information above will NOT BE graded. This assignment sheet must also be submitted along with your solution.
    % \end{enumerate}
% I agree with 
\end{flushleft}

\begin{center}
\textbf{Theory. Maximum 5 points}    
\end{center}


\begin{enumerate}
\item  \textit{Definitions, simple proofs.}

\begin{enumerate}
\item (1 point) Vectors $v_1, v_2, ..., v_n$ are linearly independent if...
\item (2 points) Give a definition of an orthogonal matrix. Provide an example.
\item (2 points) Give a definition of a normal vector to a plane?
\item (2 points) Rank of a matrix A is a ...
\item (3 points) Give definition of a trace of a matrix and prove its linearity.
\end{enumerate}

\end{enumerate}


\begin{center}
\textbf{Practice. Maximum 25 points}    
\end{center}
\begin{enumerate}
\item \textit{Vector operations / Matrices} 
\begin{enumerate}
\item (2 points) Suppose A, B and C are 3x3 matrices, and $|A|=2, |B|=3,|C|=5$. Find $|AB|-|A^{-1}B| + |2C|$.

\item (4 points) Find a vector that is orthogonal to both $v_1= (1,0,1)$ and $v_2=(1,3,0)$ and which dot product with vector $v_3=(1,1,0)$ equals to $8$.

\end{enumerate}

\item  \textit{Lines / Planes}
\begin{enumerate}
\item (2 points) Find the angle between the planes $2x - y + z = 6$, $ x + y + 2z = 3$.
% answer: $\mathbf{\pi/3}$;

\item (4 points)  What is the general equation of the plane which contains the following two parallel lines:
$\frac{x+1}{6} = \frac{y-2}{7} = z $
and 
$\frac{x-3}{6} = \frac{y+4}{7} = z -1 $
% answer: $\mathbf{13x-2y-64z+17 = 0}$;
\end{enumerate}


\item \textit{Conics}
\begin{enumerate}
\item (3 points) What are the coordinates of the vertices of the hyperbola:
$4x^2-9y^2-40x+54y-17=0$? 
% ANSWER: \textbf{(8,3) and (2,3).}

\item (5 points) 
Compose the equation of the hyperbola which axes coincide with the axes of coordinate system, if it's known that it passes through a point $(4,-2\sqrt{2})$ and touches the line $3x+y+8=0$.


\end{enumerate}

\item \textit{Coordinates / Affine Transformations}
\begin{enumerate}
\item (2 points) What is the equation of the line $y=1$ in polar coordinate system?
% \textbf{ANSWER:} $r=\frac{1}{\sin\theta}$

\item (7 points)
Determine the type of a conic $\frac{19}{4}x^2+\frac{43}{12}y^2+\frac{7\sqrt{3}}{6}xy=48$. Compose the canonical equation and the transition matrix to canonical coordinate system. Find the coordinates of its foci in the source coordinate system.
\end{enumerate}

\item  \textit{Surfaces}
\begin{enumerate}
\item (4 points) Find the radius of the circle defined by the following equations: 

$x^2 + y^2 + z^2 - 8x + 4y + 8z - 45 = 0$,  $ x - 2y + 2z = 3$.
% answer:  $\mathbf{\sqrt{80}}$;

\item (7 points) 
 Compose the equation of a right circular cone obtained by the revolution of the line $x=0,~~y-z+1=0$ around $Oz$ axis.




\end{enumerate}
\end{enumerate}


% Polar coordinates
\textbf{Extra-task. 5 points} (Polar coordinates)
If the tangents at the points P and Q on a conic intersect in T and the chord PQ meets the directrix at R then prove that the angle TSR is a right angle.
\begin{center}
End of Exam
\end{center}


\newpage

\centering
Analytical Geometry and Linear Algebra~I. Final Exam. December 14, 2022

\textbf{VARIANT 2}
\bigskip

\begin{tabular}{cc}
    \begin{tabular}[b]{|p{11cm}|c|}
    \hline
    Full name: & Group: \\ \hline
     &  \\ 
     & \\
     \hline
    \end{tabular}
\\

    % \begin{tabular}[b]{|c|c|c|c|c|c|c|c|c|c|c|c|}
    % \hline
    % Task: & 1 & 2 & 3 & 4 & 5 & 6 & 7 &8 & 9 & 10 & Total \\ \hline

    % Score: &  &  &  &  &  & &  & & &  & \_\_\_\_\_ of 30 pts. \\ \hline
    % \end{tabular}
\end{tabular}

\begin{flushleft}
    % \textbf{\Large{Rules:}}
    % \begin{enumerate}
    \textbf{In each sheet}, you \textbf{should} write your last name, first name, variant number, and group number in the \textbf{upper right} corner. Unsigned sheets or sheets without the information above will NOT BE graded. This assignment sheet must also be submitted along with your solution.
    % \end{enumerate}
% I agree with 
\end{flushleft}

\begin{center}
\textbf{Theory. Maximum 5 points}    
\end{center}


\begin{enumerate}
\item  \textit{Definitions, simple proofs.}

\begin{enumerate}
\item (1 point) Let  $V = \{v_1, v_2, ..., v_n\}$ be a set of vectors. Give a definition of a span $S(V)$
\item (2 points) A matrix is singular if ..
\item (2 points) Give a definition of a basis of vector space
\item (2 points) What is the geometrical interpretation of the magnitude of $\mathbf{a}\times \mathbf{b}$?
\item (3 points) Given that $BC$ and $CB$ are valid, prove that
 $Tr(BC) = Tr(CB)$
\end{enumerate}

\end{enumerate}


\begin{center}
\textbf{Practice. Maximum 25 points}    
\end{center}
\begin{enumerate}
\item \textit{Vector operations / Matrices} 
\begin{enumerate}
\item (2 points) 
Suppose A, B and C are 4x4 matrices, and $|A|=3, |B|=4,|C|=5$. Find $|3B|-|A^{\top}C|$ .

\item (4 points) 
Find a vector that is orthogonal to both $v_1= (1,-1,1)$ and $v_2=(6,-3,0)$ and which dot product with a vector $v_3=(3,2,3)$ equals to $10$.

% answer% (1,2,1)
\end{enumerate}

\item  \textit{Lines / Planes}
\begin{enumerate}
\item (2 points) 
Find the angle between the planes $x + y - 4 z = 8$, $4x + y - z = 8$.

% answer: pi/3

\item (4 points) 
What is the general equation of the plane which contains the following two parallel lines:
$\frac{x-1}{5} = \frac{y+2}{3} = z $
and 
$\frac{x+3}{5} = \frac{y-4}{3} = z - 1 $

% answer: ;
\end{enumerate}


\item \textit{Conics}
\begin{enumerate}
\item (3 points) What are the coordinates of the vertices of the hyperbola:

$-2x^2 + 5y^2 - 12x - 50y - 3=0$? 

%ANSWER: $(-3, 5+\sqrt(22))$ and $(-3, 5-\sqrt(22).$

\item (5 points) 
Compose the equation of the hyperbola which axes coincide with the axes of coordinate system, if it's known that it touches the lines $x=1$ and $5x-2y+3=0$.
\end{enumerate}

\item \textit{Coordinates / Affine Transformations}
\begin{enumerate}
\item (2 points) 
What is the equation of the line $x=1$ in polar coordinate system?
% \textbf{ANSWER:} $r=\frac{1}{\cos\theta}$

\item (7 points)
Determine the type of a conic $3x^2+3y^2-2xy-\frac{6}{\sqrt{2}}x-\frac{6}{\sqrt{2}}y=8$. Compose the canonical equation and transition matrix to the canonical coordinate system. Find the coordinates of its foci in the source coordinate system.
\end{enumerate}

\item  \textit{Surfaces}
\begin{enumerate}
\item (4 points) 
Find coordinates where the curve defined by
$2x^2 - 8x +8 + 2y^2=2z^2 -12z +18, z=0$ crosses the X-axis. What is the type of the surface and curve?
% answer:  $\mathbf{\sqrt{80}}$;

\item (7 points) 
 Compose the equation of a right circular cylinder passing through a point $M(1,1,2)$ and having the axis $x=1+t,~~y=2+t,~~z=3+t$.
\end{enumerate}
\end{enumerate}


\textbf{Extra-task. 5 points}
Show that the locus of the foot of the perpendicular drawn from the origin to the tangent to the circle $r= 2a \  cos \theta$ is $r = a(1 + cos \theta).$

\begin{center}
End of Exam
\end{center}
\end{document}