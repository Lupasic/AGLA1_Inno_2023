\documentclass[a4paper,10pt]{article}
%\documentclass[a4paper,10pt]{scrartcl}

\usepackage[utf8]{inputenc}
\usepackage[russian]{babel} % Cyrillics!
\usepackage{cmap} % Copypastable cyrillics!
\usepackage{amsmath} % Multiline equations
\usepackage{amssymb} % Cool symbols (like |R)
\usepackage{mathtools} % Disable numbering equations by default
\mathtoolsset{showonlyrefs}
\usepackage{indentfirst} % Indent first paragraph in chapter
\usepackage{multicol}


\makeatletter
\renewcommand*\env@matrix[1][*\c@MaxMatrixCols c]{%
  \hskip -\arraycolsep
  \let\@ifnextchar\new@ifnextchar
  \array{#1}}
\makeatother

\newcommand{\T}{\textrm{T}}
\newcommand{\I}{{-1}}

\title{Class \#4}
\author{Oleg Bulichev}
\date{}

\usepackage{color}
\newcommand{\adv}[0]{\textcolor{red}{*}}

\makeatletter
% \pdfinfo{%some PDF metadate
%   /Title    (\@title)
%   /Author   (\@author)
%   /Subject  (Essentials of Analytical Geometry and Linear Algebra I)
% }
\makeatother

\begin{document}
%\maketitle
\section*{Essentials of Analytical Geometry and Linear Algebra~I, Class \#4}
\noindent\textbf{Innopolis University, September 2022}
\\


\section{Inverse Matrix}
\begin{enumerate}
\item Find inverse matrices for the following matrices:
\begin{enumerate}

    \item $\begin{bmatrix}
     3 & 5 \\
     5 & 9 \\
   \end{bmatrix}$;
    \item $\begin{bmatrix}2&-1&0\\0&2&-1\\-1&-1&1\end{bmatrix}$;
\end{enumerate}


\item Solve matrix equations:
\begin{enumerate}

    \item $\begin{bmatrix}2&5\\1&3\end{bmatrix}X=\begin{bmatrix}2&1\\1&1\end{bmatrix}$;
    \item  $X\begin{bmatrix}2&5\\1&3\end{bmatrix}=\begin{bmatrix}2&1\\1&1\end{bmatrix}$;
\end{enumerate}

\end{enumerate}

\section{Matrix Rank}
\begin{enumerate}
\item Calculate the ranks of the following matrices:
\begin{enumerate}
    \item  $\begin{bmatrix}1&2&3\\2&3&4\\1&1&1\end{bmatrix}$.
\end{enumerate}
% \item Determine the ranks of the following matrices for all real values of parameter $\alpha$:
% \begin{enumerate}

%     \item $\begin{bmatrix}1&\alpha&-1&2\\2&-1&\alpha&5\\1&10&-6&1\end{bmatrix}$;
%     \item $\begin{bmatrix}1&1&1\\1&\alpha&\alpha^2\\1&\alpha^2&\alpha\end{bmatrix}$;
% \end{enumerate}

\end{enumerate}

\section{Changing Basis and Coordinates}
\begin{enumerate}
\item If vectors \textbf{a} and \textbf{b} form a basis (you should check it), it is needed to find coordinates $\textbf{c}$ and $\textbf{d}$ in the basis.\\\\
$\textbf{a}=\begin{bmatrix} -5 \\ -1 \end{bmatrix}$, $\textbf{b}=\begin{bmatrix} -1 \\ 3\end{bmatrix}$, 
$\textbf{c}=\begin{bmatrix} -1 \\ 2 \end{bmatrix}$, 
$\textbf{d}=\begin{bmatrix} 2 \\ -6\end{bmatrix}$.
\item There are to bases in $R^3$:

$e_1=i,\ e_2=j,\ e_3=k$ and $e_1'=i+j+k,\ e_2'=i+j,\ e_3'=i$

Find coordinates of $x=2i-3j+k$ in the basis $e_1',\ e_2',\ e_3'$.

\item

There are 4 vectors $f_1,\ f_2,\ f_3,\ x$ and the basis \\ $e_1=\begin{bmatrix}1\\0\\0\end{bmatrix},\ e_2=\begin{bmatrix}0\\1\\0\end{bmatrix},\ e_3=\begin{bmatrix}0\\0\\1\end{bmatrix}$. Find the coordinates of $x$ in the basis $(f_1,\ f_2,\ f_3)$, if $f_1=\begin{bmatrix}1\\1\\1\end{bmatrix},\ f_2=\begin{bmatrix}1\\2\\1\end{bmatrix},\ f_3=\begin{bmatrix}1\\2\\2\end{bmatrix},\ x=\begin{bmatrix}1\\1\\0\end{bmatrix}$
\end{enumerate}

\end{document}
