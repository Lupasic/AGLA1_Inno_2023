\documentclass[a4paper,10pt]{article}
%\documentclass[a4paper,10pt]{scrartcl}

\usepackage[utf8]{inputenc}
\usepackage[russian]{babel} % Cyrillics!
\usepackage{cmap} % Copypastable cyrillics!
\usepackage{amsmath} % Multiline equations
\usepackage{amssymb} % Cool symbols (like |R)
\usepackage{mathtools} % Disable numbering equations by default
\mathtoolsset{showonlyrefs}
\usepackage{indentfirst} % Indent first paragraph in chapter
\usepackage{multicol}


\makeatletter
\renewcommand*\env@matrix[1][*\c@MaxMatrixCols c]{%
  \hskip -\arraycolsep
  \let\@ifnextchar\new@ifnextchar
  \array{#1}}
\makeatother

\newcommand{\T}{\textrm{T}}
\newcommand{\I}{{-1}}

\title{Class \#12}
\author{Oleg Bulichev}
\date{}

\usepackage{color}
\newcommand{\adv}[0]{\textcolor{red}{*}}


\begin{document}
%\maketitle
\section*{Analytical Geometry and Linear Algebra~I, Class \#12}
\noindent\textbf{Innopolis University, November 2022}
\section{Polar coordinates}
\begin{enumerate}

\item Find the equation of the line joining the points $\begin{bmatrix} 2 \\ \dfrac{\pi}{3} \end{bmatrix}$ and $\begin{bmatrix} 3 \\ \dfrac{\pi}{6} \end{bmatrix}$. It should deduce that this line also passes through the point $\begin{bmatrix} \dfrac{6}{3\sqrt{3}-2} \\ \dfrac{\pi}{2} \end{bmatrix}$.

\item Find the equation of the line perpendicular to $\dfrac{l}{r}=\cos(\theta-\alpha)+e\ \cos(\theta) $ and passing through the point $\begin{bmatrix} r_1 \\ \theta_1\end{bmatrix}$.

% \textit{Ans}: $\dfrac{r_1 \sin(\theta_1 - \alpha) + e\ \sin(\theta_1)}{r}= \sin(\theta - \alpha) + e\ \sin(\theta)$

\item Show that the feet of the perpendiculars from the origin on the sides of the triangle formed by the points with vectorial angles $\alpha,\ \beta,\ \gamma$ and which lie on the circle $r = 2a \cos(\theta)$ lie on the straight line $2a\ \cos(\alpha) \cos(\beta) \cos (\gamma) = r\ \cos(\pi - \alpha - \beta - \gamma)$.

\item A focal chord $SP$ of an ellipse is inclined at an angle $\alpha$ to the major axis. Prove that the perpendicular from the focus on the tangent at $P$ makes with the axis an angle $\arctan(\dfrac{\sin(\alpha)}{e+\cos(\alpha)})$

\end{enumerate}
\end{document} 