\documentclass[a4paper,10pt]{article}
%\documentclass[a4paper,10pt]{scrartcl}

\usepackage[utf8]{inputenc}
\usepackage[russian]{babel} % Cyrillics!
\usepackage{cmap} % Copypastable cyrillics!
\usepackage{amsmath} % Multiline equations
\usepackage{amssymb} % Cool symbols (like |R)
\usepackage{mathtools} % Disable numbering equations by default
\mathtoolsset{showonlyrefs}
\usepackage{indentfirst} % Indent first paragraph in chapter
\usepackage{multicol}


\makeatletter
\renewcommand*\env@matrix[1][*\c@MaxMatrixCols c]{%
  \hskip -\arraycolsep
  \let\@ifnextchar\new@ifnextchar
  \array{#1}}
\makeatother

\newcommand{\T}{\textrm{T}}
\newcommand{\I}{{-1}}

\title{Class \#11}
\author{Oleg Bulichev}
\date{}

\usepackage{color}
\newcommand{\adv}[0]{\textcolor{red}{*}}


\begin{document}
%\maketitle
\section*{Analytical Geometry and Linear Algebra~I, Class \#11}
\noindent\textbf{Innopolis University, October 2022}
\begin{enumerate}
\item Linear transformation of a real axis is given by $f(x)=ax+b$. (a) Find all fixed points of this transformation. (b) Find the transformation that is inverse for $f$.

\textcolor{magenta}{\textbf{Answer.} (a) If $a\neq1$ then there is one fixed point $x=\frac b{1-a}$; if $a=1$ and $b=0$ then all points are fixed; if $a=1$ and $b\neq0$ then there are no fixed points. (b) It exists only if $a\neq0$: $f^{-1}(y)=\frac{y-b}a$.}
    
\item Two linear transformations of a real axis $f$ and $g$ are given by $f(x)=ax+b$, $g(x)=cx+d$. Find compositions of transformations $fg$ and $gf$. What are the necessary and sufficient conditions for $fg$ to be equal to $gf$?
    
\textcolor{magenta}{\textbf{Answer.} $(fg)(x)=acx+ad+b$; $(gf)(x)=acx+bc+d$; $fg=gf\Leftrightarrow d(a-1)=b(c-1)$.}    
    
\item Transformation of a plane is given by $x*=x^2-y^2$, $y*=2xy$. Is this transformation an (a) injection; (b) surjection; (c) bijection? (d) Find the preimage of point $(x*;y*)$ by this transformation.
    
\item Find the image of an arbitrary point $M$ which has position vector $\textbf{r}$ by the following transformations:

(a) homothety with center $M_0(\textbf{r}_0)$ and ratio $\lambda\neq0$;

(b) reflection across point $M_0(\textbf{r}_0)$;

(c) translation by vector $\textbf{a}$;

(d) orthogonal projection onto the line $\textbf{r}=\textbf{r}_0+\textbf{a}t$;

(e) reflection across the line $\textbf{r}=\textbf{r}_0+\textbf{a}t$; 

(f) dilation of factor $\lambda>0$ from the line $\textbf{r}=\textbf{r}_0+\textbf{a}t$.

\textcolor{magenta}{\textbf{Answer.} (a) $\textbf{r}*=\textbf{r}_0+\lambda(\textbf{r}-\textbf{r}_0)$; (b) $\textbf{r}*=-\textbf{r}+2\textbf{r}_0$; (c) $\textbf{r}*=\textbf{r}+\textbf{a}$; (d) $\textbf{r}*=\textbf{r}_0+\frac{\left(\textbf{r}-\textbf{r}_0\right)\cdot\textbf{a}}{|\textbf{a}|^2}\textbf{a}$; (e) $\textbf{r}*=2\textbf{r}_0-\textbf{r}+2\frac{\left(\textbf{r}-\textbf{r}_0\right)\cdot\textbf{a}}{|\textbf{a}|^2}\textbf{a}$; (f) $\textbf{r}*=\lambda\textbf{r}+(1-\lambda)\textbf{r}_0+(1-\lambda) \frac{\left(\textbf{r}-\textbf{r}_0\right)\cdot\textbf{a}}{|\textbf{a}|^2}\textbf{a}$.}

\item Images of vertices $A$, $B$, $C$ of a triangle $ABC$ by some affine transformation are midpoints $K$, $L$, $M$ of their opposite sides. Find the images by this transformation of points $K$, $L$, $M$ and centroid $O$ of triangle $ABC$. What type of transformation is it?

\textcolor{magenta}{\textbf{Answer.} Homothety with center $O$ and ratio $-0{.}5$.}

\item Prove that:

(a) if $A$ and $B$ are two fixed points of an affine transformation then all points of line $AB$ are fixed;

(b) if an affine transformation has a single fixed point then all invariant lines of this transformation pass through this point;

(c) intersection point of two invariant lines of an affine transformation is a fixed point.

\item Prove that two lines tangent to the ellipse are parallel if and only if the touching points and the center of the ellipse are collinear.
    
\item An ellipse is inscribed into a parallelogram $ABCD$ and it touches its side $AD$ at its midpoint $M$. Prove that this ellipse touches the other sides of a parallelogram in their midpoints.
    
\item An ellipse with center $O$ is inscribed into a quadrilateral $ABCD$. Prove that\footnote{$A$ means area.} $A_{\triangle OAB}+A_{\triangle OCD}=A_{\triangle OBC}+A_{\triangle OAD}$.
    
\item An affine transformation is given by $x*=3x+2y-6$, $y*=4x-3y+1$. Find the images of (a) point $M(-1;\,5)$; (b) line $2x+3y=7$.
    
\textcolor{magenta}{\textbf{Answer.} (a) $(1;-18)$; (b) $18x-5y-6=0$.}    

\item An affine transformation is given by $x*=2x+3y-1$, $y*=-3x-4y+2$. Find the preimages of (a) point $M(4;-5)$; (b) line $x+y=1$.
    
\textcolor{magenta}{\textbf{Answer.} (a) $(1;\,1)$; (b) $x+y=0$.}    
    
\item Find formulas for an affine mapping that transforms

(a) points $A(\frac37;\,1)$, $B(1;\frac14)$, $C(2;-1)$ into points $A*(-4;\,2)$, $B*(-1;\,6)$, $C*(4;\,13)$ respectively;

(b) points $A(0;\,0)$, $B(-1;\,2)$, $C(1;-2)$ into points $A*(-1;-1)$, $B*(0;\,0)$, $C*(1;\,1)$ respectively;

(c) points $A(2;\,0)$, $B(3;-1)$, $C(4;-2)$ into points $A*(2;\,1)$, $B*(-2;-1)$, $C*(-6;-3)$ respectively;

(d) points $A(-2;\,0)$, $B(2;-1)$, $C(0;\,4)$ into points $A*(-2;\,1)$, $B*(2;\,1)$, $C*(0;\,1)$ respectively.

\textcolor{magenta}{\textbf{Answer.} (a) $x*=-4y$, $y*=7x-1$; (b) no solutions; (c) $x*=px+(p+4)y+2-2p$, $y*=qx+(q+2)y+1-2q$, where $p$ and $q$ are any real numbers; (d) no solutions (there exists a linear transformation that is not affine).}

\item Find all fixed point of an affine transformation given by

(a) $x*=2x-y+3$, $y*=-2x+2y-6$;

(b) $x*=4x+3y-1$, $y*=-3x-2y+1$.

\textcolor{magenta}{\textbf{Answer.} (a) $(-3;\,0)$; (b) all points that belong to a line $3x+3y-1=0$.}

\item Find all invariant lines of an affine transformation given by

(a) $x*=y$, $y*=1-x$;

(b) $x*=2x+y-3$, $y*=-3x-y$;

(c) $x*=5x+3y+1$, $y*=-3x-y$.

\textcolor{magenta}{\textbf{Answer.} (a) no solutions; (b) $x+y-3=0$, $2x-y+p=0$, where $p$ can be any real number; (c) $x+y+1=0$.}

\item Find formulas for an affine mapping that transforms lines $x-y+1=0$ and $x+y-1=0$ into lines $3x+2y-3=0$ and $2x+3y+1=0$ respectively and point $A(1;\,1)$ into point $B(-1;-2)$.
    
\textcolor{magenta}{\textbf{Answer.} $x*=-\frac{16}5x+\frac{44}5y-\frac{33}5$, $y*=-\frac15x-\frac{41}5y+\frac{32}5$.}    
    
\item Find formulas for an affine transformation such that hyperbola $\frac{x^2}5-\frac{y^2}4=1$ is invariant under this transformation and image of $A(5;\,4)$ is $B(\sqrt5;\,0)$.

\textcolor{magenta}{\textbf{Answer.} $x*=\sqrt5(x-y)$, $y*=\pm\sqrt5\left(\frac{4x}5-y\right)$.}

\item Find formulas for the following affine transformations:

(a) orthogonal projection onto line $x-3y+1=0$;

(b) reflection across line $3x+4y-1=0$;

(c) dilation from line $x+y-2=0$ of factor $\frac13$;

(d) dilation from line $2x-y+5=0$ of factor 2.   

\textcolor{magenta}{\textbf{Answer.} (a) $x*=\frac{9x+3y-1}{10}$, $y*=\frac{3x+y+3}{10}$; (b) $x*=\frac{7x-24y+6}{25}$, $y*={-24x-7y+8}{25}$; (c) $x*=\frac{2x-y+2}3$, $y*=\frac{-x+2y+2}3$; (d) $x*=\frac{9x-2y+10}5$, $y*=\frac{-2x+6y-5}5$.}
\end{enumerate}    
\end{document} 